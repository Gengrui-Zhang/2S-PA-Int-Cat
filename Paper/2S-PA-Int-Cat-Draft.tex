% Options for packages loaded elsewhere
\PassOptionsToPackage{unicode}{hyperref}
\PassOptionsToPackage{hyphens}{url}
%
\documentclass[
  man]{apa6}
\usepackage{amsmath,amssymb}
\usepackage{iftex}
\ifPDFTeX
  \usepackage[T1]{fontenc}
  \usepackage[utf8]{inputenc}
  \usepackage{textcomp} % provide euro and other symbols
\else % if luatex or xetex
  \usepackage{unicode-math} % this also loads fontspec
  \defaultfontfeatures{Scale=MatchLowercase}
  \defaultfontfeatures[\rmfamily]{Ligatures=TeX,Scale=1}
\fi
\usepackage{lmodern}
\ifPDFTeX\else
  % xetex/luatex font selection
\fi
% Use upquote if available, for straight quotes in verbatim environments
\IfFileExists{upquote.sty}{\usepackage{upquote}}{}
\IfFileExists{microtype.sty}{% use microtype if available
  \usepackage[]{microtype}
  \UseMicrotypeSet[protrusion]{basicmath} % disable protrusion for tt fonts
}{}
\makeatletter
\@ifundefined{KOMAClassName}{% if non-KOMA class
  \IfFileExists{parskip.sty}{%
    \usepackage{parskip}
  }{% else
    \setlength{\parindent}{0pt}
    \setlength{\parskip}{6pt plus 2pt minus 1pt}}
}{% if KOMA class
  \KOMAoptions{parskip=half}}
\makeatother
\usepackage{xcolor}
\usepackage{graphicx}
\makeatletter
\newsavebox\pandoc@box
\newcommand*\pandocbounded[1]{% scales image to fit in text height/width
  \sbox\pandoc@box{#1}%
  \Gscale@div\@tempa{\textheight}{\dimexpr\ht\pandoc@box+\dp\pandoc@box\relax}%
  \Gscale@div\@tempb{\linewidth}{\wd\pandoc@box}%
  \ifdim\@tempb\p@<\@tempa\p@\let\@tempa\@tempb\fi% select the smaller of both
  \ifdim\@tempa\p@<\p@\scalebox{\@tempa}{\usebox\pandoc@box}%
  \else\usebox{\pandoc@box}%
  \fi%
}
% Set default figure placement to htbp
\def\fps@figure{htbp}
\makeatother
\setlength{\emergencystretch}{3em} % prevent overfull lines
\providecommand{\tightlist}{%
  \setlength{\itemsep}{0pt}\setlength{\parskip}{0pt}}
\setcounter{secnumdepth}{-\maxdimen} % remove section numbering
% Make \paragraph and \subparagraph free-standing
\makeatletter
\ifx\paragraph\undefined\else
  \let\oldparagraph\paragraph
  \renewcommand{\paragraph}{
    \@ifstar
      \xxxParagraphStar
      \xxxParagraphNoStar
  }
  \newcommand{\xxxParagraphStar}[1]{\oldparagraph*{#1}\mbox{}}
  \newcommand{\xxxParagraphNoStar}[1]{\oldparagraph{#1}\mbox{}}
\fi
\ifx\subparagraph\undefined\else
  \let\oldsubparagraph\subparagraph
  \renewcommand{\subparagraph}{
    \@ifstar
      \xxxSubParagraphStar
      \xxxSubParagraphNoStar
  }
  \newcommand{\xxxSubParagraphStar}[1]{\oldsubparagraph*{#1}\mbox{}}
  \newcommand{\xxxSubParagraphNoStar}[1]{\oldsubparagraph{#1}\mbox{}}
\fi
\makeatother
% definitions for citeproc citations
\NewDocumentCommand\citeproctext{}{}
\NewDocumentCommand\citeproc{mm}{%
  \begingroup\def\citeproctext{#2}\cite{#1}\endgroup}
\makeatletter
 % allow citations to break across lines
 \let\@cite@ofmt\@firstofone
 % avoid brackets around text for \cite:
 \def\@biblabel#1{}
 \def\@cite#1#2{{#1\if@tempswa , #2\fi}}
\makeatother
\newlength{\cslhangindent}
\setlength{\cslhangindent}{1.5em}
\newlength{\csllabelwidth}
\setlength{\csllabelwidth}{3em}
\newenvironment{CSLReferences}[2] % #1 hanging-indent, #2 entry-spacing
 {\begin{list}{}{%
  \setlength{\itemindent}{0pt}
  \setlength{\leftmargin}{0pt}
  \setlength{\parsep}{0pt}
  % turn on hanging indent if param 1 is 1
  \ifodd #1
   \setlength{\leftmargin}{\cslhangindent}
   \setlength{\itemindent}{-1\cslhangindent}
  \fi
  % set entry spacing
  \setlength{\itemsep}{#2\baselineskip}}}
 {\end{list}}
\usepackage{calc}
\newcommand{\CSLBlock}[1]{\hfill\break\parbox[t]{\linewidth}{\strut\ignorespaces#1\strut}}
\newcommand{\CSLLeftMargin}[1]{\parbox[t]{\csllabelwidth}{\strut#1\strut}}
\newcommand{\CSLRightInline}[1]{\parbox[t]{\linewidth - \csllabelwidth}{\strut#1\strut}}
\newcommand{\CSLIndent}[1]{\hspace{\cslhangindent}#1}
\ifLuaTeX
\usepackage[bidi=basic]{babel}
\else
\usepackage[bidi=default]{babel}
\fi
\babelprovide[main,import]{english}
% get rid of language-specific shorthands (see #6817):
\let\LanguageShortHands\languageshorthands
\def\languageshorthands#1{}
\ifLuaTeX
  \usepackage[english]{selnolig} % disable illegal ligatures
\fi
% Manuscript styling
\usepackage{upgreek}
\captionsetup{font=singlespacing,justification=justified}

% Table formatting
\usepackage{longtable}
\usepackage{lscape}
% \usepackage[counterclockwise]{rotating}   % Landscape page setup for large tables
\usepackage{multirow}		% Table styling
\usepackage{tabularx}		% Control Column width
\usepackage[flushleft]{threeparttable}	% Allows for three part tables with a specified notes section
\usepackage{threeparttablex}            % Lets threeparttable work with longtable

% Create new environments so endfloat can handle them
% \newenvironment{ltable}
%   {\begin{landscape}\centering\begin{threeparttable}}
%   {\end{threeparttable}\end{landscape}}
\newenvironment{lltable}{\begin{landscape}\centering\begin{ThreePartTable}}{\end{ThreePartTable}\end{landscape}}

% Enables adjusting longtable caption width to table width
% Solution found at http://golatex.de/longtable-mit-caption-so-breit-wie-die-tabelle-t15767.html
\makeatletter
\newcommand\LastLTentrywidth{1em}
\newlength\longtablewidth
\setlength{\longtablewidth}{1in}
\newcommand{\getlongtablewidth}{\begingroup \ifcsname LT@\roman{LT@tables}\endcsname \global\longtablewidth=0pt \renewcommand{\LT@entry}[2]{\global\advance\longtablewidth by ##2\relax\gdef\LastLTentrywidth{##2}}\@nameuse{LT@\roman{LT@tables}} \fi \endgroup}

% \setlength{\parindent}{0.5in}
% \setlength{\parskip}{0pt plus 0pt minus 0pt}

% Overwrite redefinition of paragraph and subparagraph by the default LaTeX template
% See https://github.com/crsh/papaja/issues/292
\makeatletter
\renewcommand{\paragraph}{\@startsection{paragraph}{4}{\parindent}%
  {0\baselineskip \@plus 0.2ex \@minus 0.2ex}%
  {-1em}%
  {\normalfont\normalsize\bfseries\itshape\typesectitle}}

\renewcommand{\subparagraph}[1]{\@startsection{subparagraph}{5}{1em}%
  {0\baselineskip \@plus 0.2ex \@minus 0.2ex}%
  {-\z@\relax}%
  {\normalfont\normalsize\itshape\hspace{\parindent}{#1}\textit{\addperi}}{\relax}}
\makeatother

\makeatletter
\usepackage{etoolbox}
\patchcmd{\maketitle}
  {\section{\normalfont\normalsize\abstractname}}
  {\section*{\normalfont\normalsize\abstractname}}
  {}{\typeout{Failed to patch abstract.}}
\patchcmd{\maketitle}
  {\section{\protect\normalfont{\@title}}}
  {\section*{\protect\normalfont{\@title}}}
  {}{\typeout{Failed to patch title.}}
\makeatother

\usepackage{xpatch}
\makeatletter
\xapptocmd\appendix
  {\xapptocmd\section
    {\addcontentsline{toc}{section}{\appendixname\ifoneappendix\else~\theappendix\fi: #1}}
    {}{\InnerPatchFailed}%
  }
{}{\PatchFailed}
\makeatother
\keywords{keywords\newline\indent Word count: X}
\DeclareDelayedFloatFlavor{ThreePartTable}{table}
\DeclareDelayedFloatFlavor{lltable}{table}
\DeclareDelayedFloatFlavor*{longtable}{table}
\makeatletter
\renewcommand{\efloat@iwrite}[1]{\immediate\expandafter\protected@write\csname efloat@post#1\endcsname{}}
\makeatother
\usepackage{lineno}

\linenumbers
\usepackage{csquotes}
\usepackage{bookmark}
\IfFileExists{xurl.sty}{\usepackage{xurl}}{} % add URL line breaks if available
\urlstyle{same}
\hypersetup{
  pdftitle={2S-PA-Int-Cat},
  pdfauthor={Gengrui (Jimmy) Zhang1},
  pdflang={en-EN},
  pdfkeywords={keywords},
  hidelinks,
  pdfcreator={LaTeX via pandoc}}

\title{2S-PA-Int-Cat}
\author{Gengrui (Jimmy) Zhang\textsuperscript{1}}
\date{}


\shorttitle{2S-PA-Int-Cat}

\authornote{

The authors made the following contributions. Gengrui (Jimmy) Zhang: Conceptualization, Writing - Original Draft Preparation, Writing - Review \& Editing.

Correspondence concerning this article should be addressed to Gengrui (Jimmy) Zhang. E-mail: \href{mailto:gengruiz@email.com}{\nolinkurl{gengruiz@email.com}}

}

\affiliation{\vspace{0.5cm}\textsuperscript{1} University of Southhern California}

\abstract{%
Two-stage path analysis with interaction for categorical variables.
}



\begin{document}
\maketitle

\section{Methods}\label{methods}

We adopted a fully crossed design with varying conditions of sample
size, composite reliability of scale, interaction effect, and item
skewness, based on the study design of Aytürk et al.~(2020) and
(\textbf{hsiaoModelingMeasurementErrors2021?}). We compared the accuracy of UPI
with three product--indicator formations (all-pair, matched-pair, and
parceled-pair), LMS for categorical items, and 2S-PA-Int in recovering
the latent interaction effect.

\subsection{Population Structural Model}\label{population-structural-model}

We considered a latent regression model with two exogenous latent
variables for person \(j\) with \(j=1,\ldots,N\), \(\xi_{x_{j}}\) and
\(\xi_{_{j}}\), and one endogenous latent outcome, \(\xi_{_{j}}\), as our
population model. The primary estimand was the standardized latent
interaction effect of \(\xi_{x_{j}}\) and \(\xi_{m_{j}}\) on \(\xi_{y_{j}}\),
denoted \(\gamma_{xm}\):

\begin{align}
\intertext{For $j=1,\ldots,N$,}
\begin{bmatrix}\xi_{x_{j}} \\ \xi_{m_{j}}\end{bmatrix}
&\stackrel{\text{i.i.d.}}{\sim}
\mathcal{N}\!\left(
\begin{bmatrix}0\\[2pt]0\end{bmatrix},
\begin{bmatrix}1 & \rho \\ \rho & 1\end{bmatrix}
\right),\quad
\xi_{y_{j}} = \alpha + \gamma_{x}\,\xi_{x_{j}} + \gamma_{m}\,\xi_{m_{j}}
             + \gamma_{xm}\,\xi_{x_{j}}\xi_{m_{j}} + \zeta_{j},
\end{align}

where \(\alpha\) was the constant intercept set to 1.2. \(\xi_{x_{j}}\) was
the first-order latent predictor with a fixed main effect
\(\gamma_{x} \;=\; 0.3\), and \(\xi_{m_{j}}\) was the first-order latent
moderator also with a fixed main effect \(\gamma_{m} \;=\; 0.3\). Both
exogenous factors were factors were standardized with zero means and
unit variances. Additionally, they were pre-specified with a fixed
correlation of \(\rho \;=\; 0.3\), and they were allowed to freely correlated with the latent interaction term. We examined two population values of
the latent interaction effect: \(\gamma_{xm} \;=\; 0\) to test the null
hypothesis (\(H_{0}\)) and \(\gamma_{xm} \;=\; 0.3\) (medium effect; Cohen,
1992) to test the alternative hypothesis (\(H_{1}\)). The variance of
latent outcome variable \(\xi_{y_{j}}\) was set to 1 when
\(\gamma_{xm} \;=\; 0\) under the \(H_{0}\) condition, so that the variance
of disturbance term
\(\sigma^2_{\zeta} \;=\; 1 - \big(\gamma_x^2 + \gamma_m^2 + \gamma_{xm}^2 + 2\,\gamma_x\gamma_m\,\rho \big)\).
Therefore, the values of \(\sigma^2_{\zeta}\) were adjusted according to
the latent interaction effect size (e.g.,
\(\sigma_\zeta^2 = 1 - \left(0.3^2 + 0.3^2 + 0 + 2 \times 0.3 \times 0.3 \times 0.3\right) = 0.766\)
when \(\gamma_{xm} = 0\), which indicated that the first-order latent
predictors and the latent interaction term jointly contributed to
explain \(23.3\%\) variance in \(\xi_{y_{j}}\).

\subsection{Population Measurement Model}\label{population-measurement-model}

After drawing the person-level latent scores from the structural model,
we generated observed indicators for each construct. Let \(i\) index items with \(j=1,\ldots,K\). The latent outcome variable \(\xi_{y_j}\) was measured by three continuous indicators \((y_{1j},y_{2j},y_{3j})\) with differential loadings:
\begin{align}
y_{ij} \;=\; \lambda_{y_i}\,\xi_{y_j} \;+\; \delta_{y_{ij}},
\qquad i = 1,2,3,
\end{align}
where \(\lambda_{y}\)s were sepcified as \(\{.50,\ .70,\ .90\}\), and the error variances followed
\(\delta_{y_{ij}}\sim\mathcal{N}\!\big(0,\;1-\lambda_{y_i}^{2}\big)\).

Following the design in Aytürk et al.~(2020), we set \(\xi_{j}\) with three items (i.e., \(x_{1j},\ldots,x_{2j}\)) and \(\xi_{m}\) with twelve items (i.e., \(m_{1j},\ldots,m_{12j}\)). We first formed continuous precursors with varying item loadings,
\begin{align}
x^{\ast}_{ij} &= \lambda_{x_i}\,\xi_{x_j} + \delta_{x_{ij}}, 
& i=1,2,3, \\
m^{\ast}_{ij} &= \lambda_{m_i}\,\xi_{m_j} + \delta_{m_{ij}}, 
& i=1,\ldots,12,
\end{align}
where \(x_{ij}^*\) was the score of the underlying latent continuous variable for each observed categorical item \(i\). \(\delta_{x_{ij}}\) was the individual-specific error term for each
observed indicator \(i\) which followed a normal distribution with a zero mean and a variance \(\theta_{x_{ij}}\).

\subsubsection{\texorpdfstring{Categories of Observed Indicators for \(\xi_{x}\) and}{Categories of Observed Indicators for \textbackslash xi\_\{x\} and}}\label{categories-of-observed-indicators-for-xi_x-and}

Observed categorical indicators (e.g., \(x_{ij}\) for \(\xi_{x_{j}}\)) were
generated from a normal--ogive (cumulative probit) graded response model
with differential factor loadings (Cho, 2023). For person \(j\) and item
\(i\) measuring latent factor \(\eta_x\), \begin{equation}
x_{ij}^* = \lambda_{x_{i}}\xi_{x_{j}} + \delta_{x_{ij}},
\qquad \delta_{x_{ij}} \sim \mathcal{N}(0,1),
\end{equation} where \(x_{ij}^*\) is the score of underlying latent
continuous variable for each observed categorical item \(i\).
\(\delta_{x_{ij}}\) is the individual-specific error term for each
observed indicator \(i\), which follows a standard normal distribution.
Given \(x_{ij}^*\), the observed categorical item \(x_{ij}\) can be created
with multiple categories through thresholding: \begin{equation}
  x_{ij} =
    \begin{cases}
      0 & \text{if $x_{ij}^* < \beta_{x_{i1}}$}\\
      k & \text{if $\beta_{x_{ik}} \le x_{ij}^* < \beta_{x_{i(k + 1)}}$}\\
      K - 1 & \text{if $\beta_{x_{i(K - 1)}} \le x_{ij}^*$}
    \end{cases},      
\end{equation} where \(\beta_{ik}\) is the threshold parameter between the
\(k\)th and \((k + 1)\)th category for \(k = 1, 2,...,K\). The observed items
for \(\xi_{m_{j}}\) (e.g., \(m_{ij}\)) were generated analogously.
\(\{\tau_{m_i,k}\}\).

\subsection{Simulation Design}\label{simulation-design}

\subsection{Material}\label{material}

\subsection{Procedure}\label{procedure}

\subsection{Data analysis}\label{data-analysis}

We used R (Version 4.5.1; R Core Team, 2025) and the R-packages \emph{papaja} (Version 0.1.3; Aust \& Barth, 2024) and \emph{tinylabels} (Version 0.2.5; Barth, 2025) for all our analyses.

\section{Results}\label{results}

\section{Discussion}\label{discussion}

\newpage

\section{References}\label{references}

\phantomsection\label{refs}
\begin{CSLReferences}{1}{0}
\bibitem[\citeproctext]{ref-R-papaja}
Aust, F., \& Barth, M. (2024). \emph{{papaja}: {Prepare} reproducible {APA} journal articles with {R Markdown}}. \url{https://doi.org/10.32614/CRAN.package.papaja}

\bibitem[\citeproctext]{ref-R-tinylabels}
Barth, M. (2025). \emph{{tinylabels}: Lightweight variable labels}. \url{https://doi.org/10.32614/CRAN.package.tinylabels}

\bibitem[\citeproctext]{ref-R-base}
R Core Team. (2025). \emph{R: A language and environment for statistical computing}. Vienna, Austria: R Foundation for Statistical Computing. Retrieved from \url{https://www.R-project.org/}

\end{CSLReferences}


\end{document}
